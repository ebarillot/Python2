\documentclass{beamer}
% spécification encodage indispensable
\usepackage[utf8]{inputenc}
% nécessaire pour éviter un warnong sur l'usage du french et de OT1; pas compris pourquoi fondamentalement
\usepackage[T1]{fontenc}
% pour les partilularités du français, par exemple les guillements
\usepackage[french]{babel}
% nécessaire pour la commande \mathbb{R} pour l'ensemble R
\usepackage{amsfonts}


\usetheme{Warsaw}
\title{Licence de Mathématiques \\3M248 - Projet}
\author{Emmanuel Barillot - 3370161}
\date{09 mai 2018}

\begin{document}
\begin{frame}
\titlepage
\end{frame}


\begin{frame}
	\frametitle{Origine des données}
	\begin{itemize}
		\item Fichier de données INSEE de 2014 sur les 36000 communes françaises
		\item traitement des données brutes
		\begin{itemize}
			\item suppression des variables non retenues
			\item agrégation des données par département
			\item séparation des variables en plusieurs familles
			\begin{itemize}
				\item salaires
				\item démographie
				\item économie
			\end{itemize}
		\end{itemize}

		\item Idée: brosser un portrait des départements en fonction des trois aspects
		\item Après essais, une variable a semblé intéressante: la ruralité
			\begin{itemize}
				\item[$\rightarrow$] calculée à partir du nb de communes de moins de $10000$ hab dans chaque département
			\end{itemize}
	\end{itemize}
\end{frame}


\begin{frame}
	\frametitle{Variables}
	\begin{itemize}
	\item variables relatives aux salaires: 
		\begin{itemize}
		\item Moyenne Revenus Fiscaux Départementaux,
	    \item Moyenne Salaires Cadre Horaires,
	    \item Moyenne Salaires Prof Intermédiaire Horaires,
	    \item Moyenne Salaires Employé Horaires,
	    \item Moyenne Salaires Ouvrier Horaires,
		\end{itemize}
	\item variables relatives à la population d'entreprises et d'organismes:
		\begin{itemize}
		\item Nb Education, santé, action sociale,
		\item Nb Entreprises,
		\item Nb Création Entreprises,
		\end{itemize}
	\item variables relatives à la population:
		\begin{itemize}
		\item Population,
		\item Evolution Population,
		\item Nb Ménages,
		\item Nb propriétaire,
		\item Nb Etudiants.
		\end{itemize}
	\end{itemize}
\end{frame}


\begin{frame}
	\frametitle{Démographie}
	\framesubtitle{Boxplot}
	\begin{center}
	\includegraphics[scale=0.4]{pop_boxplots.eps}
	\end{center}
\end{frame}


\begin{frame}
	\frametitle{Démographie}
	\framesubtitle{Scatter}
	\begin{center}
	\includegraphics[scale=0.4]{pop_grid_scatter.eps}
	\end{center}
\end{frame}


\begin{frame}
	\frametitle{Démographie}
	\framesubtitle{Scatter 3D}
	\begin{center}
	\includegraphics[scale=0.4]{pop_1_scatter3D.eps}
	\end{center}
\end{frame}


\begin{frame}
	\frametitle{Démographie}
	\framesubtitle{ACP}
	\begin{center}
	\includegraphics[scale=0.4]{pop_2CP.eps}
	\end{center}
\end{frame}


\begin{frame}
	\frametitle{Démographie}
	\framesubtitle{Cerle des corrélations}
	\begin{center}
	\includegraphics[scale=0.4]{pop_Corr.eps}
	\end{center}
\end{frame}


\begin{frame}
	\frametitle{Démographie}
	\framesubtitle{Variance}
	\begin{center}
	\includegraphics[scale=0.4]{pop_Var.eps}
	\end{center}
\end{frame}


\begin{frame}
	\frametitle{Démographie}
	\framesubtitle{Classification: K-means}
	\begin{center}
	\includegraphics[scale=0.4]{pop_Kmeans.eps}
	\end{center}
\end{frame}


\begin{frame}
	\frametitle{Démographie}
	\framesubtitle{Regroupement hiérarchique ascendant}
	\begin{center}
	\includegraphics[scale=0.4]{pop_rha_tree.eps}
	\end{center}
\end{frame}


%\begin{frame}
%	\frametitle{Entreprises}
%	\framesubtitle{Regroupement hiérarchique ascendant}
%	\begin{center}
%	\includegraphics[scale=0.4]{entrep_rha_bar.eps}
%	\end{center}
%\end{frame}
%
%
%\begin{frame}
%	\frametitle{Entreprises}
%	\framesubtitle{Boxplot}
%	\begin{center}
%	\includegraphics[scale=0.4]{entrep_boxplots.eps}
%	\end{center}
%\end{frame}
%
%
%\begin{frame}
%	\frametitle{Entreprises}
%	\framesubtitle{Scatter}
%	\begin{center}
%	\includegraphics[scale=0.4]{entrep_grid_scatter.eps}
%	\end{center}
%\end{frame}
%
%
%\begin{frame}
%	\frametitle{Entreprises}
%	\framesubtitle{Scatter 3D}
%	\begin{center}
%	\includegraphics[scale=0.4]{entrep_1_scatter3D.eps}
%	\end{center}
%\end{frame}
%
%
%\begin{frame}
%	\frametitle{Entreprises}
%	\framesubtitle{ACP}
%	\begin{center}
%	\includegraphics[scale=0.4]{entrep_2CP.eps}
%	\end{center}
%\end{frame}
%
%
%\begin{frame}
%	\frametitle{Entreprises}
%	\framesubtitle{Cerle des corrélations}
%	\begin{center}
%	\includegraphics[scale=0.4]{entrep_Corr.eps}
%	\end{center}
%\end{frame}
%
%
%\begin{frame}
%	\frametitle{Entreprises}
%	\framesubtitle{Variance}
%	\begin{center}
%	\includegraphics[scale=0.4]{entrep_Var.eps}
%	\end{center}
%\end{frame}
%
%
%\begin{frame}
%	\frametitle{Entreprises}
%	\framesubtitle{Classification: K-means}
%	\begin{center}
%	\includegraphics[scale=0.4]{entrep_Kmeans.eps}
%	\end{center}
%\end{frame}
%
%
%\begin{frame}
%	\frametitle{Entreprises}
%	\framesubtitle{Regroupement hiérarchique ascendant}
%	\begin{center}
%	\includegraphics[scale=0.4]{entrep_rha_tree.eps}
%	\end{center}
%\end{frame}
%
%
%\begin{frame}
%	\frametitle{Entreprises}
%	\framesubtitle{Regroupement hiérarchique ascendant}
%	\begin{center}
%	\includegraphics[scale=0.4]{entrep_rha_bar.eps}
%	\end{center}
%\end{frame}

\end{document}

