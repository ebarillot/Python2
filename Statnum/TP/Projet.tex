\documentclass[a4paper,11pt]{article}
% spécification encodage indispensable
\usepackage[utf8]{inputenc}
% nécessaire pour éviter un warnong sur l'usage du french et de OT1; pas compris pourquoi fondamentalement
\usepackage[T1]{fontenc}
% pour les partilularités du français, par exemple les guillements
\usepackage[french]{babel}
% nécessaire pour la commande \mathbb{R} pour l'ensemble R
\usepackage{amsfonts}

% pour régler les marges
\usepackage{geometry}
\geometry{hmargin=3cm,vmargin=2cm}

% pour les énumérations
\usepackage{enumerate}

% pour avoir le 1 de la fonction caractéristique affichée comme R
% avec un double barre
\usepackage{bbm}

% pour les lettres majuscules rondes
\usepackage{mathrsfs}

% pour les URLs
\usepackage{hyperref}

% voir aide Latex ici : https://fr.sharelatex.com/


%\usepackage[scr]{rsfso}
%\usepackage{lmodern}
%\usepackage{amsthm}
%\usepackage{amssymb}
\usepackage{amsmath}
%\usepackage{latexsym}
%\usepackage{eepic}
%\usepackage{epsfig}
%\usepackage{graphicx}
%\usepackage{amscd}
%\usepackage{mathtools}
%\usepackage{csquotes}
%%\usepackage[margin=3cm]{geometry}
%\usepackage[text={16cm,22.9cm,centering}]{geometry}
%\usepackage{upgreek}
%\usepackage{enumitem}


\title{Licence de Mathématiques \\3M248 - Projet}


\author{Emmanuel Barillot - 3370161}

% pour supprimer l'indentation de la 1ere ligne ce chaque paragraphe
\setlength{\parindent}{0pt}

\begin{document}
\maketitle



\section{Introduction}
Notre jeu de données provient des bases de données de l'INSEE. Il se compose d'un fichier Excel qui contient des informations sur les 36000 (environ) communes françaises.
Il a été publié le 12 décembre 2014 à l'adresse: 
\href{https://www.data.gouv.fr/s/resources/data-insee-sur-les-communes/20141212-105948/MDB-INSEE-V2.xls}{https://www.data.gouv.fr/s/resources/data-insee-sur-les-communes/20141212-105948/MDB-INSEE-V2.xls}

Nous nous proposons d'analyser ces données de façon à chercher des liens entre les caractéristiques démographiques et les caractéristiques économiques des départements français.

\section{Présentation des données}
Le fichier brut contient plus de 36000 lignes et 100 colonnes.
Certaines variables ne nous sont pas utiles comme le code de la région.
Chaque ligne est identifiée par le code postal de la commune.
Il y a des variables qualitatives que nous jugeons peu exploitables, comme l'aspect rural ou urbain d'une commune (deux valeurs disponibles).
Certaines variables quantitatives sont relatives au bassin de vie, dont la définition n'est pas donnée dans ce jeu de données.
Certaines variables semblent redondantes, nous n'avons gardé que celle qui nous semblait la plus pertinente, sans plus d'explications de la part de l'INSEE.
Chaque ligne contient le code du département auquel elle appartient.
Nous allons concentrer notre étude sur les 100 département présents dans le fichier, en procédant par agrégation des variables quantitatives.


Nous procédons d'abord à une transformation simple des données brutes:
\begin{itemize}
\item suppression des variables non retenues
\item agrégation des données par département
\item séparation des variables en deux familles: caractéristiques démographiques et caractéristiques économiques.
\end{itemize}

Nous obtenons finalement les deux jeux de données suivants:


\subsection{Exploration}
\subsection{Analyse multivariée}
\subsection{Classification}

\section{Conclusion}



\end{document}

